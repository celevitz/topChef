\nonstopmode{}
\documentclass[a4paper]{book}
\usepackage[times,inconsolata,hyper]{Rd}
\usepackage{makeidx}
\makeatletter\@ifl@t@r\fmtversion{2018/04/01}{}{\usepackage[utf8]{inputenc}}\makeatother
% \usepackage{graphicx} % @USE GRAPHICX@
\makeindex{}
\begin{document}
\chapter*{}
\begin{center}
{\textbf{\huge Package `topChef'}}
\par\bigskip{\large \today}
\end{center}
\ifthenelse{\boolean{Rd@use@hyper}}{\hypersetup{pdftitle = {topChef: Top Chef Data}}}{}
\ifthenelse{\boolean{Rd@use@hyper}}{\hypersetup{pdfauthor = {Levitz Carly E}}}{}
\begin{description}
\raggedright{}
\item[Type]\AsIs{Package}
\item[Title]\AsIs{Top Chef Data}
\item[Version]\AsIs{0.1.0}
\item[Author]\AsIs{Carly E Levitz [aut,cre]}
\item[Maintainer]\AsIs{Carly E Levitz }\email{celevitz@gmail.com}\AsIs{}
\item[Description]\AsIs{Several datasets which describe the chef contestants in Top Chef, 
the challenges that they compete in, and the results of those challenges. 
This data is useful for practicing data wrangling, graphing, and analyzing 
how each season of Top Chef played out.}
\item[URL]\AsIs{}\url{https://github.com/celevitz/topChef}\AsIs{}
\item[BugReports]\AsIs{}\url{https://github.com/celevitz/topChef/issues}\AsIs{}
\item[Depends]\AsIs{R (>= 4.2.0)}
\item[Imports]\AsIs{tidyr (>= 1.3.0), dplyr (>= 1.1.0)}
\item[License]\AsIs{MIT + file LICENSE}
\item[Encoding]\AsIs{UTF-8}
\item[LazyData]\AsIs{true}
\item[RoxygenNote]\AsIs{7.3.1}
\item[Suggests]\AsIs{knitr, rmarkdown, testthat (>= 3.0.0), magrittr (>= 2.0.0),
ggplot2 (>= 3.4.0)}
\item[Config/testthat/edition]\AsIs{3}
\item[VignetteBuilder]\AsIs{knitr}
\item[Language]\AsIs{en-US}
\end{description}
\Rdcontents{Contents}
\HeaderA{topChef-package}{topChef: Top Chef Data}{topChef.Rdash.package}
\aliasA{topChef}{topChef-package}{topChef}
\keyword{internal}{topChef-package}
%
\begin{Description}
Several datasets which describe the chef contestants in Top Chef, the challenges that they compete in, and the results of those challenges. This data is useful for practicing data wrangling, graphing, and analyzing how each season of Top Chef played out.
\end{Description}
%
\begin{Author}
\strong{Maintainer}: Levitz Carly E \email{celevitz@gmail.com} [copyright holder]

\end{Author}
%
\begin{SeeAlso}
Useful links:
\begin{itemize}

\item{} \url{https://github.com/celevitz/topChef}
\item{} Report bugs at \url{https://github.com/celevitz/topChef/issues}

\end{itemize}


\end{SeeAlso}
\HeaderA{challengedescriptions}{challengedescriptions}{challengedescriptions}
\keyword{datasets}{challengedescriptions}
%
\begin{Description}
A dataset containing information about each challenge that the
Chefs compete in
\end{Description}
%
\begin{Usage}
\begin{verbatim}
data(challengedescriptions)
\end{verbatim}
\end{Usage}
%
\begin{Format}
This data frame contains the following columns:
\begin{description}

\item[\code{season}] Name of season
\item[\code{seasonNumber}] Season number
\item[\code{series}] Top Chef US (listed as US); Top Chef US Masters
(listed as US Masters); Top Chef Canada (listed as Canada)
\item[\code{episode}] Episode number
\item[\code{challengeType}] Challenge type: qualifying challenge,
elimination, quickfire, sudden death quickfire, quickfire
elimination, battle of the sous chefs
\item[\code{outcomeType}] Is the challenge run as a team or as an
individual?
\item[\code{challengeDescription}] Description of the challenge
\item[\code{shopTime}] If they go shopping, how long do they have?
Unit is minutes
\item[\code{shopBudget}] If they go shopping, what is their budget?
Unit is dollars unless otherwise specified.
\item[\code{prepTime}] If they have prep time, how long do they have?
Unit is minutes
\item[\code{cookTime}] How long they have to cook (in minutes)
\item[\code{productPlacement}] List of products promoted in the
challenge, other than the usual series-wide product placement.
Will be blank if none were mentioned
\item[\code{advantage}] If an advantage is offered to the winner of the
challenge, it will be listed here: e.g., Immunity, choosing
a protein in the elimination challenge, choosing your team in
the elimination challenge. Will be blank if none were mentioned.
\item[\code{lastChanceKitchenWinnerEnters}] If someone comes in from
Last Chance Kitchen at this challenge, their name will be listed here.
Will be blank for all other challenges.
\item[\code{restaurantWarWinner}] Role played by the winner of
restaurant wars: Executive Chef, Front of House, the full team,
Line Cook, Roles Rotated, or No one won. Will only have values
for Restaurant War episodes.
\item[\code{restaurantWarEliminated}] Role played by the Chef eliminated
after restaurant wars: Executive Chef, Front of House, the full
team, Line Cook, Roles Rotated. Will only have values for
Restaurant War episodes.
\item[\code{didJudgesVisitWinningTeamFirst}] Categorical variable of
which team was shown serving the judges first. Will only have values for
Restaurant Wars episodes.

\end{description}

\end{Format}
%
\begin{Source}
\url{https://en.wikipedia.org/wiki/Top_Chef}
\end{Source}
%
\begin{Examples}
\begin{ExampleCode}
library(dplyr)
library(tidyr)
challengedescriptions %>%
   group_by(series,season,outcomeType) %>%
   summarise(n=n()) %>%
   pivot_wider(names_from=outcomeType,values_from=n)
\end{ExampleCode}
\end{Examples}
\HeaderA{challengewins}{challengewins}{challengewins}
\keyword{datasets}{challengewins}
%
\begin{Description}
A dataset containing win and loss data for each chef in each episode
\end{Description}
%
\begin{Usage}
\begin{verbatim}
data(challengewins)
\end{verbatim}
\end{Usage}
%
\begin{Format}
This data frame contains the following columns:
\begin{description}

\item[\code{season}] Name of season
\item[\code{seasonNumber}] Season number
\item[\code{series}] Top Chef US (listed as US); Top Chef US Masters
(listed as US Masters); Top Chef Canada (listed
as Canada)
\item[\code{episode}] Episode number
\item[\code{inCompetition}] True / false for whether the Chef was still
in the competition at the time of the
challenge
\item[\code{immune}] True / false for whether that Chef was immune from
being eliminated for challenge
\item[\code{chef}] Name of chef
\item[\code{challengeType}] Challenge type: qualifying challenge,
elimination, quickfire, sudden death quickfire,
quickfire elimination, battle of the sous
chefs
\item[\code{outcome}] Result for each Chef in the competition for that
challenge
\item[\code{rating}] Numeric rating provided to chefs in Top Chef US
Masters Seasons 1 and 2. Will be blank for all
other seasons.

\end{description}

\end{Format}
%
\begin{Source}
\url{https://en.wikipedia.org/wiki/Top_Chef}
\end{Source}
%
\begin{Examples}
\begin{ExampleCode}
library(dplyr)
library(tidyr)
challengewins %>%
  group_by(outcome) %>%
  summarise(n=n())
\end{ExampleCode}
\end{Examples}
\HeaderA{chefdetails}{chefdetails}{chefdetails}
\keyword{datasets}{chefdetails}
%
\begin{Description}
A dataset containing information on each Chef for each season. As of now,
it has data for all Top Chef US seasons, Top Chef Masters (US), and one
season of Top Chef Canada.
\end{Description}
%
\begin{Usage}
\begin{verbatim}
data(chefdetails)
\end{verbatim}
\end{Usage}
%
\begin{Format}
This data frame contains the following columns:
\begin{description}

\item[\code{name}] Chef name (full name)
\item[\code{chef}] Shorter version of the chef's name
\item[\code{hometown}] Chef's hometown, if known
\item[\code{city}] City in which the Chef lived at the time of show
\item[\code{state}] State in which the Chef lived at the time of the show
\item[\code{age}] Age of Chef at the time of the show
\item[\code{season}] Name of season
\item[\code{seasonNumber}] Season number
\item[\code{series}] Top Chef US (listed as US); Top Chef US Masters
(listed as US Masters); Top Chef Canada (listed
as Canada)
\item[\code{placement}] Final result of the Chef.
\item[\code{personOfColor}] Flag for whether the Chef is a person of color.
Will be blank if they are not
\item[\code{occupation}] Occupation of Chef at time of show, if known
\item[\code{occupation\_category}] Categorization of occupation
\item[\code{gender}] Gender of Chef

\end{description}

\end{Format}
%
\begin{Source}
\url{https://en.wikipedia.org/wiki/Top_Chef}
\end{Source}
%
\begin{Examples}
\begin{ExampleCode}
library(dplyr)
library(tidyr)
chefdetails %>%
  filter(season == "World All Stars")
\end{ExampleCode}
\end{Examples}
\HeaderA{episodeinfo}{episodeinfo}{episodeinfo}
\keyword{datasets}{episodeinfo}
%
\begin{Description}
A dataset containing information about each episode
\end{Description}
%
\begin{Usage}
\begin{verbatim}
data(episodeinfo)
\end{verbatim}
\end{Usage}
%
\begin{Format}
This data frame contains the following columns:
\begin{description}

\item[\code{season}] Name of season
\item[\code{seasonNumber}] Season number
\item[\code{series}] Top Chef US (listed as US); Top Chef US Masters
(listed as US Masters); Top Chef Canada (listed as
Canada)
\item[\code{overallEpisodeNumber}] Running number of episode within
the series
\item[\code{episode}] Episode number
\item[\code{episodeName}] Name of episode
\item[\code{airDate}] Date the episode originally aired
\item[\code{nCompetitors}] Number of Chefs still in the competition

\end{description}

\end{Format}
%
\begin{Source}
\url{https://en.wikipedia.org/wiki/Top_Chef}
\end{Source}
%
\begin{Examples}
\begin{ExampleCode}
library(dplyr)
library(tidyr)
episodeinfo %>% filter(season=="World All Stars")
\end{ExampleCode}
\end{Examples}
\HeaderA{judges}{judges}{judges}
\keyword{datasets}{judges}
%
\begin{Description}
A dataset containing information about who were the guest judges for
each challenge
\end{Description}
%
\begin{Usage}
\begin{verbatim}
data(judges)
\end{verbatim}
\end{Usage}
%
\begin{Format}
This data frame contains the following columns:
\begin{description}

\item[\code{season}] Name of season
\item[\code{seasonNumber}] Season number
\item[\code{series}] Top Chef US (listed as US); Top Chef US Masters
(listed as US Masters); Top Chef Canada (listed as Canada)
\item[\code{episode}] Episode number
\item[\code{challengeType}] Challenge type: qualifying challenge,
elimination, quickfire, sudden death quickfire, quickfire
elimination, battle of the sous chefs
\item[\code{outcomeType}] Is the challenge run as a team or as an
individual?
\item[\code{guestJudge}] Name of guest judge
\item[\code{gender}] Gender of Chef
\item[\code{personOfColor}] Flag for whether the Chef is a person of color.
Will be blank if they are not
\item[\code{competedOnTC}] Will have a value of Yes if they competed
on a season of Top Chef
\item[\code{otherShows}] Information about other shows that this
individual has appeared on

\end{description}

\end{Format}
%
\begin{Source}
\url{https://en.wikipedia.org/wiki/Top_Chef}
\end{Source}
%
\begin{Examples}
\begin{ExampleCode}
library(dplyr)
library(tidyr)
judges %>%
  filter(guestJudge == "Eric Ripert") %>%
  group_by(challengeType) %>%
  summarise(n=n())
\end{ExampleCode}
\end{Examples}
\HeaderA{rewards}{rewards}{rewards}
\keyword{datasets}{rewards}
%
\begin{Description}
A dataset containing information about rewards and prizes won by challenge
\end{Description}
%
\begin{Usage}
\begin{verbatim}
data(rewards)
\end{verbatim}
\end{Usage}
%
\begin{Format}
This data frame contains the following columns:
\begin{description}

\item[\code{season}] Name of season
\item[\code{seasonNumber}] Season number
\item[\code{series}] Top Chef US (listed as US); Top Chef US Masters
(listed as US Masters); Top Chef Canada (listed as Canada)
\item[\code{episode}] Episode number
\item[\code{challengeType}] Challenge type: qualifying challenge,
elimination, quickfire, sudden death quickfire, quickfire elimination,
battle of the sous chefs
\item[\code{outcomeType}] Is the challenge run as a team or as an
individual?
\item[\code{rewardType}] Variable describing whether the reward is
money or a prize
\item[\code{reward}] Description of the full reward
\item[\code{chef}] Name of chef

\end{description}

\end{Format}
%
\begin{Source}
\url{https://en.wikipedia.org/wiki/Top_Chef}
\end{Source}
%
\begin{Examples}
\begin{ExampleCode}
library(dplyr)
library(tidyr)
rewards %>%
  filter(rewardType == "Money") %>%
  mutate(reward=as.numeric(reward)) %>%
  group_by(season) %>%
  summarise(total=sum(reward))
\end{ExampleCode}
\end{Examples}
\HeaderA{weightedindex}{Calculate One Season's Chef's Weighted Scores (Index)}{weightedindex}
%
\begin{Description}
Calculates the Index score for each person within a season of
Top Chef.
\end{Description}
%
\begin{Usage}
\begin{verbatim}
weightedindex(
  seriesname,
  seasonnumberofchoice,
  numberofelimchalls,
  numberofquickfires
)
\end{verbatim}
\end{Usage}
%
\begin{Arguments}
\begin{ldescription}
\item[\code{seriesname}] Values can be: US, US Masters, Canada

\item[\code{seasonnumberofchoice}] Integer of the season number within the series

\item[\code{numberofelimchalls}] Number of elimination challenges you want to
include in the index. Must be greater than 0

\item[\code{numberofquickfires}] Number of quickfire challenges you want to include
in the index. Must be greater than 0.
\end{ldescription}
\end{Arguments}
%
\begin{Details}
The result of elimination challenges and quickfire challenges are
weighted.
Scoring: Elimination win = +7 points. Elimination high = +3 points.
Elimination low = -3 points. Eliminated = -7 points.
Quickfire win = +4 points. Quickfire high = +2 points.
Quickfire low = -2 points.
Combines Sudden Death Quickfires with Eliminations
Excludes Qualifying Challenges
Holding constant the number of elimination challenges and quickfire
challenges will allow comparison across seasons if you want
\end{Details}
%
\begin{Value}
Tibble of index score for each contestant in that season and their
placement
\end{Value}
\printindex{}
\end{document}
